\documentclass{article}
\addtolength{\oddsidemargin}{-.875in}
\addtolength{\evensidemargin}{-.875in}
\addtolength{\textwidth}{1.75in}
\addtolength{\topmargin}{-1.0in}
\addtolength{\textheight}{1.80in}
\usepackage{wrapfig}
\usepackage{sidecap}
\usepackage[pdftex]{graphicx}
\usepackage[utf8]{inputenc}
\usepackage[spanish]{babel}
\begin{document}
\pagestyle{empty}
\noindent

\noindent
{\sc Parcial \#1  - Física II (2014-20)---Profesor: Jaime Forero--- 27
de Agosto del 2014}

NOTA: Todas las respuestas deben tener una justificaci\'on f\'isica y
matem\'atica de acuerdo a los temas vistos en el curso. Se permite el
uso de calculadora.


%{\bf Instrucciones}\\
%\begin{itemize}
%\item
%Lea completo el ex\'amen antes de empezar a responderlo. Hay 2 p\'aginas y 7 ejercicios. Pregunte en voz alta si tiene dudas sobre alg\'un enunciado.
%\item
%El ex\'amen tiene 120 puntos posibles  como m\'aximo y 0.0 puntos como m\'inimo%. 100 puntos corresponden a una nota de 5.0
%\item
%Para escribir las respuestas y  los c\'alculos auxiliares, solamente se pueden utilizar las hojas blancas que reciben al principio. Cada hoja debes estar marcada con nombre y c\'odigo. Las hojas de borrador se deben entregar al final.
%\item SI se permite el uso de calculadora.
%\item
%El uso de tel\'efonos celulares, laptops, tabletas, etc durante un examen ser\'a considerado como fraude. Esta falta ser\'a reportada ante el comit\'e disciplinario de la Facultad de Ciencias para entablar el proceso correspondiente seg\'un el reglamento de pregrado. 
%\end{itemize}




  Problema 1. (15 puntos). Para enfriar $800$ g de agua caliente a 90 $^\circ$C,
  se agrega un pedazo de hielo de $100$ g que se encuentra a $-30$
  $^{\circ}$C. Despu\'es de que el sistema llega al equilibrio
  t\'ermico ¿Se derrite todo el hielo? El calor espec\'ifico
  del agua l\'iquida es $4190$ J kg$^{-1}$ K$^{-1}$, del agua
  s\'olida es $2144$ J kg$^{-1}$ K$^{-1}$ y el calor latente de
  fusi\'on del agua es $334000$ J/kg. 

\

Problema 2. (15 puntos). El calor espec\'ifico del aluminio es un quinto del calor espec\'ifico
del agua. Si 1kg de aluminio a 90$^\circ$C se sumergen en 4kg de agua
a $5^\circ$C. ¿Cu\'al es la temperatura de equilibrio?  

\



Problema 3. (15 puntos). 
Un recipiente de volumen fijo contiene 2 moles de un gas ideal
monoat\'omico a una temperatura de 27$^{\circ}$C. Un segundo
recipiente con el mismo volumen tiene el doble de cantidad de moles de
un gas ideal diat\'omico. La presi\'on en ambos recipientes es
inicialmente la misma. Cada uno de estos recipientes recibe el mismo
flujo de calor constante y su temperatura empieza a aumentar. ¿Cu\'al
de los dos gases alcanza primero una temperatura de 527$^{\circ}$C?     

\

Problema 4. (15 puntos). Si el volumen y la temperatura de un gas ideal
cambian de tal manera que el volumen final es tres veces el valor inicial y
su temperatura final es un cuarto de la temperatura inicial. ¿C\'omo cambia el n\'umero de colisiones por unidad de tiempo de las mol\'eculas contra las paredes del recipiente? 

\

Problema 5. (20 puntos) Una mol de Hidr\'ogeno diat\'omico tiene una
temperatura de $0^{\circ}$ y se calienta a presi\'on constante. La
presi\'on es desconocida.\\
%
a) (10 puntos). ¿Cu\'anto calor hay que transferirle al gas para doblar su volumen?\\
b) (10 puntos). ¿Cu\'anto vale el trabajo producido por el gas en ese caso?\\


\begin{wrapfigure}{l}{0.5\textwidth}
%\centering
\begin{center}
\includegraphics[width=0.40\textwidth]{fig5OttoIdeal_web.jpg}
\end{center}
\caption{Ciclo Otto, problema 6.}
\end{wrapfigure} 


\

Problema 6 (20 puntos). El ciclo Otto se muestra en la Figura 1. El
sistema contiene una mol de gas ideal y pasa por 5 puntos diferentes
en el plano $PV$ con temperaturas $T_1$, $T_2$, $T_3$, $T_4$ y
$T_5$.
  Las curvas corresponden a procesos adiab\'aticos. \footnote{Imagen
    tomada de
    \verb"http://web.mit.edu/16.unified/www/SPRING/propulsion/notes/node25.html"}. \\
a) (3 puntos). Encuentre la cantidad de calor que entra ($Q_H$) en t\'erminos de las
  temperaturas $T_2$ y $T_3$.\\
b)(3 puntos). Encuentre la cantidad de calor que sale ($Q_L$) en t\'erminos de las
  temperaturas $T_1$ y $T_4$.\\
c) (2 puntos). La eficiencia de este ciclo se define como $e =
  1+\frac{Q_H}{Q_{L}}$. Encuentre la eficiencia $e$ en t\'erminos de las
  temperaturas $T_1$, $T_2$, $T_3$ y $T_4$.\\
d) (12 puntos). Sabiendo que las curvas en el plano PV son adiab\'aticas,
  exprese la eficiencia $e$ en t\'erminos de los vol\'umenes $V_1$ y
  $V_2$. \\



\

Problema 7 (20 puntos).\\
a) (16 puntos) Encuentre la capacidad calor\'ifica molar de un gas
ideal monoat\'omico que se expande seg\'un la ley $PV^{q}=constante$,
donde $q>0$.\\
b) (2 puntos) ¿Para cu\'ales valores de $q$ el calor espec\'ifico molar es infinito?\\
c) (2 puntos) ¿Para cu\'ales valores de $q$ el calor espec\'ifico molar es igual a cero?  \\


\end{document}



\item 

\item 




\item 



\end{enumerate}



\begin{enumerate}

\item 
\item
\begin{itemize}
\end{itemize}
\end{enumerate}








\end{document}
