\documentclass{article}
\addtolength{\oddsidemargin}{-.875in}
\addtolength{\evensidemargin}{-.875in}
\addtolength{\textwidth}{2.0in}
\addtolength{\topmargin}{-1.5in}
\addtolength{\textheight}{2.80in}
\usepackage{wrapfig}
\usepackage{sidecap}
\usepackage[pdftex]{graphicx}
\usepackage[utf8]{inputenc}
\usepackage[spanish]{babel}
\begin{document}
\pagestyle{empty}
\noindent

\noindent
{\sc Supletorio Parcial \#3  - Física II (2014-20)---Profesor: Jaime Forero--- 19 de Noviembre 2014}

\vspace{1cm}

NOTA: Todas las respuestas deben tener una justificaci\'on f\'isica y
matem\'atica de acuerdo a los temas vistos en el curso. Se permite el
uso de calculadora. En todo el parcial tome
$1/4\pi\epsilon_0=k=1\times 10^{10}$ N C$^{-2}$ m$^{2}$. 

\begin{enumerate}
\item (20 puntos) El campo el\'ectrico en la superficie de una esfera
  de cobre con carga, s\'olida  con radio de 10 cm  es de 2000 N/C,
  dirigido hacia afuera. ¿Cuál es el potencial (en Volts) a una distancia
  de 30cm desde el centro de la esfera si se considera que el potencial
  en el infinito es igual a cero?. 

\item (20 puntos) Un cilindro muy grande de 2cm de radio tiene una
  densidad de carga uniforme de 1.50 nC/m. Alrededor de este cilindro
  se ubica otro cilindro, con el mismo eje de simetr\'ia, de radio 5cm
  y densidad de carga uniforme de  -1.50 nC/m. 

\begin{itemize}
\item (5 puntos) Grafique de manera esquem\'atica el campo el\'ectrico radial en funci\'on de la distancia al eje de simetr\'ia de los cilindros.
\item (15 puntos)
Calcule la diferencia de
  potencial entre los dos cilindros en unidades de Volts.
\end{itemize}

\item (20 puntos) Cuanto vale la diferencia de potencial en las placas
  extremas de de un sistema compuesto por tres planos infinitos
  paralelos cargados con densidades de carga $\sigma_1$, $\sigma_2$ y
  $\sigma_3$? La placa intermedia se encuentra a una distancia
  $h_1$de la primera y a una distancia $h_2$ de la tercera. 

\item (20 puntos) Un alambre uniforme de resistencia R se corta en
  cinco piezas de igual longitud. Dos se doblan en círculo y se
  conectan entre las otras tres como se ilustra en la figura ¿Cuál es
  la resistencia entre los  extremos opuestos $a$ y $b$?

\item  En el circuito que se ilustra en la figura
  encuentre:

\begin{enumerate}
\item[a)] (7 puntos) La corriente en el resistor de 3.0$\Omega$. 
\item[b)] (7 puntos) Las fem desconocidas $\varepsilon_1$, $\varepsilon_2$.
\item[c)] (6 puntos) El valor de la resistencia $R$. 
\end{enumerate}
{\bf NOTA: La corriente que baja por la resistencia de 2.0$\Omega$ es en realidad de $5$A.}

\item (20 puntos) El circuito que se ve en la figura se conoce
  como el puente de Wheatstone, se utiliza para determinar la
  resistencia de un  resitor desconocido $X$ por comparaci\'on con
  tres resistores $M$,   $N$ y $P$ cuyas resistencias se puden
  modificar. Si el amper\'imetro $A$ no mide ning\'un flujo de
  corriente entre los puntos $a$ y $b$   cuando $M=200\Omega$,
  $N=15\Omega$ y $P=30\Omega$. ¿Cuánto vale la resistencia
  desconocida $X$?  
\end{enumerate}

\begin{center}
\includegraphics[width=0.85\textwidth]{figuras.jpg}
\end{center}

\end{document}
