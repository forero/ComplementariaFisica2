\documentclass{article}
\textheight=25.5cm
\textwidth=16.0cm
\oddsidemargin=-0.5cm
\topmargin=-3.0cm
\usepackage[pdftex]{graphicx}
\usepackage[utf8]{inputenc}
\usepackage[spanish]{babel}
\title{Taller \# 5 de F\'isica 2\\ FISI 1028, Semestre 2014 - 20}
\author{Profesor: Jaime Forero}
\date{Viernes, 29 de Agosto, 2014}
\begin{document}
\maketitle
\thispagestyle{empty}

\noindent
Este taller debe ser preprarado y discutido para la clase
complementaria de la semana del 1 de Septiembre del 2014.

Las respuestas a los cuatro primeros ejercicios se deben entregar al comenzar la
clase complementaria. Los \'ultimos dos ejercicios son para participaci\'on en clase y entrega durante la complementaria. 

\begin{enumerate}

\item
Ejercicio 21.5 (Atracci\'on entre dos personas).

\item
Ejercicio 21.10 (Masa vs. carga)

\item
Ejercicio 21.11 (Un experimento en el espacio).

\item
Ejercicio 21.24 (Tres cargas en equilibrio)


\item
Tres cargas positivas, $q_1$, $q_2$ y $q_3$ est\'an unidas entre ellas
por dos hilos. La longitud de cada hilo es $l$. Encuentre la fuerza de
tensi\'on de cada hilo.

\item
\label{exo:hex}
Siete cargas id\'enticas $q$ est\'an unidas la unas a las otras por
cuerdas el\'asticas como se muestra en la Figura \ref{fig:hex}. La
distancia entre dos cargas vecinas es $l$. Encuentre la fuerza de
tensión de todas las cuerdas. 
\end{enumerate}


\begin{figure}[h!]
\begin{center}
\includegraphics[scale=0.20]{figs/hexagono.pdf}
\end{center}
\caption{Diagrama para el problema \ref{exo:hex}}
\label{fig:hex}
\end{figure}


\end{document}
