\documentclass{article}
\textheight=25.5cm
\textwidth=16.0cm
\oddsidemargin=-0.5cm
\topmargin=-3.0cm
\usepackage[pdftex]{graphicx}
\usepackage[utf8]{inputenc}
\usepackage[spanish]{babel}
\title{Taller \# 6 de F\'isica 2\\ FISI 1028, Semestre 2014 - 20}
\author{Profesor: Jaime Forero}
\date{Viernes, 5 de Septiembre 2014}
\begin{document}
\maketitle
\thispagestyle{empty}

\noindent

Este taller debe ser preprarado y discutido para la clase
complementaria de la semana del 8 de Septiembre del 2014.

Las respuestas a los ocho primeros ejercicios se deben entregar al comenzar la
clase complementaria. Los \'ultimos cuatro ejercicios son para participaci\'on en clase y entrega durante la complementaria. 


\begin{enumerate}

\item
Ejercicio 21.33 (Electr\'on entre dos placas).

\item
Ejercicio 21.35 (Rapidez del electr\'on entre dos placas).

\item
Ejercicio 21.48 (Campo El\'ectrico de dos cargas puntuales).

\item
Ejercicio 21.57 (Campo El\'ectrico entre dos l\'aminas).

\item
Ejercicio 21.62 (L\'ineas de campo el\'ectrico de tres cargas).

\item 
Ejercicio 21.69 (Par de torsi\'on sobre un dipolo).

\item
Ejercicio 22.4 (Flujo de campo a trav\'es de un cubo).

\item 
Ejercicio 22.11 (Ley de Gauss en un medio homog\'eneo).

\item 
Problema 21.89 (carga puntual y varillla cargada I).

\item
Problema 21.90 (carga puntual y varilla cargada II).

\item 
Problema 21.96 (carga en el centro de una semi-anillo).

\item
Problema 21.103 (dos l\'aminas infinitas).
\end{enumerate}

\end{document}
