\documentclass{article}
\textheight=25.5cm
\textwidth=16.0cm
\oddsidemargin=-0.5cm
\topmargin=-3.0cm
\usepackage[pdftex]{graphicx}
\usepackage[utf8]{inputenc}
\usepackage[spanish]{babel}
\title{Taller \# 10 de F\'isica 2\\ FISI 1028, Semestre 2014 - 20}
\author{Profesor: Jaime Forero}
\date{Viernes 10 de Octubre, 2014}
\begin{document}
\maketitle
\thispagestyle{empty}

\noindent

Este taller debe ser preprarado y discutido para la clase
complementaria de la semana del 13 de Octubre del 2014.


Las respuestas a los 5 primeros ejercicios se deben entregar al comenzar la
clase complementaria. Los \'ultimos 4 ejercicios son para
participaci\'on en clase y entrega durante la complementaria. 

\begin{enumerate}

\item
Ejercicio 25.31 (Ca\'ida de potencial en un cable).

\item 
Ejercicio 25.34 (Lectura de un amper\'imetro ideal en paralelo).

\item 
Ejercicio 25.35 (Lectura de un volt\'iemtro ideal en serie).

\item 
Ejercicio 25.43 (Potencia en bombillas el\'ectricas).

\item
Ejercicio 26.1 (Resistencia de un alambre en tres piezas).

\item 
Ejercicio 26.6 (Circuito con FEM desconocida)

\item
Ejercicio 26.21 (Reglas de Kirchhoff)

\item
Ejercicio 26.23 (Reglas de Kirchhoff)

\item
Ejercicio 26.29 (Reglas de Kirchhoff)



\end{enumerate}

\end{document}
