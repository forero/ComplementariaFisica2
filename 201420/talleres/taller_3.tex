\documentclass{article}
\textheight=25.5cm
\textwidth=16.0cm
\oddsidemargin=-0.5cm
\topmargin=-3.0cm
\usepackage[pdftex]{graphicx}
\usepackage[utf8]{inputenc}
\usepackage[spanish]{babel}
\title{Taller \#3 de F\'isica 2\\ FISI 1028, Semestre 2014 - 20}
\author{Profesor: Jaime Forero}
\date{Viernes, 15 de Agosto, 2014}
\begin{document}
\maketitle
\thispagestyle{empty}

\noindent
Este taller debe ser preprarado y discutido para la clase
complementaria de la semana del 18 de Agosto del 2014. .

Las respuestas a los seis primeros ejercicios se deben entregar al comenzar la
clase complementaria. Los \'ultimos tres ejercicios son para
participac\'on en clase y entrega al final de la complementaria. 

\begin{enumerate}

\item
Ejercicio 19.20 (Trabajo en una compresi\'on isot\'ermica.)


\item 
Ejercicio 19.35 (Energ\'ia interna en un proceso adiab\'atico.)

\item 
Ejercicio 19.39 (Expansi\'on adiab\'atica de aire.)

\item 
Ejercicio 19.43 (Ciclos en un sistema arbitrario).

\item
Ejercicio 19.48 (Trabajo total en un ciclo
isob\'arico-adiab\'atico-isoc\'orico).

\item 
Ejercicio 19.51 (Cambio de volumen en un sistema no ideal).

\item
  Para subir de $1$ K la temperatura de $1$ kg de un gas desconocido a
  presi\'on constante se necesitan $912$ J. Para calentarlo a volumen
  constante se necesitan $649$ J. ¿Cuál es el gas desconocido?
  

\item Una cantidad $Q$ de calor fluye hacia un gas ideal con
  exponente adiab\'atico $\gamma$. ¿Qu\'e fracci\'on de la energ\'ia
  calor\'ifica es utilizada para efectuar el trabajo de expansi\'on
  del gas, asumiendo que el proceso ocurre a presi\'on constante? Es
  decir, ¿cu\'anto vale $W/Q$ para este proceso? Pista: La respuesta debe
  estar expresada en t\'erminos de $\gamma$ \'unicamente.  (Tomado del
  primer parcial del 2014-10.)

\item
  Un pist\'on de masa $M$ tiene encerrado un volumen $V_0$ de un gas
  monoat\'omico a presi\'on $P_0$ y temperatura $T_0$. El pist\'on se
  mueve inicialmente con una velocidad $u$ y comprime el gas. Suponiendo
  que no hay fugas de calor y despreciando las capacidades calor\'ificas
  del pistón y el recipiente, determine la temperatura y el volumen
  del gas en su compresi\'on m\'axima.


\end{enumerate}
\end{document}
