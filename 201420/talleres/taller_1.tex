\documentclass{article}
\textheight=25.5cm
\textwidth=16.0cm
\oddsidemargin=-0.5cm
\topmargin=-3.0cm
\usepackage[pdftex]{graphicx}
\usepackage[utf8]{inputenc}
\usepackage[spanish]{babel}
\title{Taller \#1 de F\'isica 2\\ FISI 1028, Semestre 2014 - 20}
\author{Profesor: Jaime Forero}
\date{Viernes, 1 de Agosto, 2014}
\begin{document}
\maketitle
\thispagestyle{empty}

\noindent
Este taller corresponde a la clase complementaria de la semana
del 4 de Agosto del 2014. Las respuestas a los cinco primeros
ejercicios (tomados del texto gu\'ia del curso) se deben entregar al
comenzar la clase complementaria. Los \'ultimos cuatro se deben
trabajar en clase y entregar al final. 

\begin{enumerate}
\item{Ejercicio 17.11. Temperatura del nitr\'ogeno l\'iquido.}

\item{Ejercicio 17.28. Expansi\'on t\'ermica de un pist\'on.} 

\item{Ejercicio 17.34. Calentando caf\'e.}

\item{Ejercicio 17.44. Estimar calores latentes de fusi\'on.}

\item{Ejercicio 17.52. Quemaduras de vapor vs. quemaduras de agua.} 


\item{Una tetera de aluminio de $2.0$kg que contiene $1.0$kg de agua se pone
  en una estufa. Pensando que no se transfiere calor al entorno. ¿Cuánto
  calor debe agregarse para elevar la temperatura de 20.0$^{\circ}$C a
  90$^{\circ}$? Tome el calor espec\'ifico del Alumino como
  $C_{Al}=910$ J kg$^{-1}$ K$^{-1}$ y el calor espec\'ifico del agua
  como $C_{H20}=4190$ J kg$^{-1}$ K$^{-1}$. (Ejercicio tomado del primer parcial
  del 2013-20.)}

\item La evaporaci\'on del sudor es un mecanismo
  importante para bajar la temperatura de algunos animales. ¿Qué masa
  de agua es mayor, la que requiere para evaporarse en la piel de un
  caballo de 500 kg para enfríar su cuerpo de  50$^{\circ}$C a
  30$^{\circ}$? ¿O la que se requiere para enfriar un caballo de 300 kg
  de 50$^{\circ}$C a 15$^{\circ}$C? Asuma que el calor de
  vaporizaci\'on del agua y el calor espec\'ifico de un caballo son
  constantes entre 0$^{\circ}$C y 80$^{\circ}$. (Ejercicio tomado del primer
  parcial del 2013-20.)  
  
\item Para enfriar $400$ g de agua caliente a 70 $^\circ$C,
  se agrega un pedazo de hielo de $10$ g que se encuentra a $-10$
  $^{\circ}$C. Despu\'es de que el sistema llega al equilibrio
  t\'ermico ¿Se
  derrite todo el hielo? El calor espec\'ifico
  del agua l\'iquida es $4190$ J kg$^{-1}$ K$^{-1}$, del agua
  s\'olida es $2144$ J kg$^{-1}$ K$^{-1}$ y el calor latente de
  fusi\'on del agua es $334000$ J/kg. (Ejercicio tomado del primer
  parcial del 2014-10).  

\item{Si el sal\'on ML617 estuviera aislado t\'ermicamente estime
  cuanto deber\'ia aumentar la temperatura durante una hora de clase
  de F\'isica II si todos los 80 alumnos est\'an presentes. Pistas para
  la soluci\'on: ¿Cu\'antos Watts netos produce una estudiante promedio?
  ¿Cu\'anto es la masa de aire dentro del sal\'on? ¿Cuánto vale el
  calor específico del aire?}

\end{enumerate}
\end{document}
