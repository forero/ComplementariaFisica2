\documentclass{article}
\textheight=25.5cm
\textwidth=16.0cm
\oddsidemargin=-0.5cm
\topmargin=-3.0cm
\usepackage[pdftex]{graphicx}
\usepackage[utf8]{inputenc}
\usepackage[spanish]{babel}
\title{Taller \#2 de F\'isica 2\\ FISI 1028, Semestre 2014 - 20}
\author{Profesor: Jaime Forero}
\date{Viernes, 8 de Agosto, 2014}
\begin{document}
\maketitle
\thispagestyle{empty}

\noindent
Este taller corresponde a la clase complementaria de la semana
del 11 de Agosto del 2014. Las respuestas a los seis primeros
ejercicios (tomados del texto gu\'ia del curso) se deben entregar al
comenzar la clase complementaria. Los \'ultimos cuatro se deben
trabajar en clase y entregar al final. 

\begin{enumerate}

\item Ejercicio 18.22. (Masa molar de un compuesto org\'anico).

\item Ejercicio 18.39 (Rapidez rms de nitr\'ogeno y hidr\'ogeno).

\item Ejercicio 18.50 (Nubes)

\item Ejercicio 19.1 (Gr\'afica PV y trabajo)

\item Ejercicio 19.8 (Energ\'ia interna y primera ley)

\item Ejercicio 19.17 (Trabajo en procesos c\'iclicos) (Problema del
  primer parcial del 2013-20).

\item
Un gas con temperatura uniforme está confinado en una mitad de una
cámara doble aislada (no hay transferencia de energía con el ambiente)
a través de una membrana. La otra mitad de la cámara se evacúa
(quedando solo vacío). Si la membrana se rompe y el gas se expande
libremente para llenar toda la doble cámara, ¿cuánto valen $Q$,
$\Delta U$ y $W$ para este proceso?  (Problema del primer
parcial del 2014-10) 

\item
Cu\'al es la energ\'ia interna (en Joules) de $1$ litro de aire en
condiciones normales de presi\'on y temperatura ($1$ atm=$10^5$ Pa).
Asuma que el aire est\'a compuesto de nitr\'ogeno molecular $N_2$ y
que la masa molar del Nitr\'ogeno monoat\'omico es $14$g. (Problema del
primer parcial del 2014-10).

\item 
Considere dos recipientes de un mismo volumen cada uno conteniendo las
mismas moles de un gas ideal monoat\'omico. La presi\'on en un
recipiente es $P_{0}$ mientras que en el otro recipiente es
$4P_{0}$. Si la velocidad $v_{\rm rms}$ de los \'atomos en el primer
recipiente es $v_{0}$, ¿cuánto es la velocidad $v_{\rm rms}$ en el segundo
recipiente? (problema del primer parcial del 2013-20)

\item
Bas\'andose en el modelo cin\'etico-molecular de un gas ideal, estime el n\'umero de mol\'eculas de aire que golpean 1 cm$^2$ de muro de su habitaci\'on en un segundo. 

\end{enumerate}
\end{document}
