\documentclass{article}
\textheight=25.5cm
\textwidth=16.0cm
\oddsidemargin=-0.5cm
\topmargin=-3.0cm
\usepackage[pdftex]{graphicx}
\usepackage[utf8]{inputenc}
\usepackage[spanish]{babel}
\title{Taller \#4 de F\'isica 2\\ FISI 1028, Semestre 2014 - 20}
\author{Profesor: Jaime Forero}
\date{Viernes, 22 de Agosto, 2014}
\begin{document}
\maketitle
\thispagestyle{empty}

\noindent
Este taller debe ser preprarado y discutido para la clase
complementaria de la semana del 25 de Agosto del 2014. .

Las respuestas a los seis primeros ejercicios se deben entregar al comenzar la
clase complementaria. Los \'ultimos cuatro ejercicios son para
participac\'on en clase y entrega al final de la complementaria. 

\begin{enumerate}


\item
Ejercicio 20.5 (Planta Nuclear.)

\item
Ejercicio 20.15 (Ciclo de Carnot.)

\item 
Ejercicio 20.22 (Máquina de Carnot que derrite hielo.)

\item 
Ejercicio 20.26 (Entropía en una bañera.)

\item 
Ejercicio 20.40 (Máquina térmica de tres pasos.)

\item 
Ejercicio 20.60 (Diagrama TS)

\item 
Encuentre el cambio de entrop\'ia de una mol de gas ideal que se expande de manera libre doblando su volumen. (Pista: El proceso de expansi\'on libre {\bf no es reversible}. Deben encontrar un proceso reversible que tenga las mismas condiciones inciales y finales.)

\item
Encuentre el cambio de entropía de $1$ kg de hielo al derretirse.

\item
Encuentre el cambio de entropía de una masa $m$ de gas ideal que pasaa de un volumen $V_1$ y temperatura $T_1$ a un volumen $V_2$ y temperatura $T_2$ en los siguientes tres casos diferentes: a) Se calienta a volumen constante $V_1$ y luego se expande de manera isoterma; b) Se expande a temperatura constante $T_1$ hasta un volumen $V_2$ y luego se caliente a volumen constante; c) se expande adiabáticamente hasta el volumen $V_2$ y luego se calienta a volumen constante.

\item
Mostrar que el rendimiento de una máquina térmica es máximo en un proceso cíclico en el cual la entropía del sistema no varía.

\end{enumerate}

\end{document}

