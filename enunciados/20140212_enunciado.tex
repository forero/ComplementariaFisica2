\documentclass{article}
\textheight=25.5cm
\textwidth=16.0cm
\oddsidemargin=-0.5cm
\topmargin=-3.0cm
\usepackage[pdftex]{graphicx}
\usepackage[utf8]{inputenc}
\usepackage[spanish]{babel}
\title{Taller \# 4 de F\'isica 2\\ FISI 1028, Semestre 2014 - 10}
\author{Profesor: Jaime Forero}
\date{Mi\'ercoles, 12 de Febrero, 2014}
\begin{document}
\maketitle
\thispagestyle{empty}

\noindent
Este taller debe ser preprarado y discutido para la clase
complementaria de la semana del 17 de Febrero del 2014. 

\noindent
{\bf Importante}:
\begin{itemize}

\item
Las respuestas a los seis primeros ejercicios se deben entregar al comenzar la
clase complementaria. 
\item 
Los \'ultimos cuatro ejercicios son para participaci\'on en clase y entrega
durante la complementaria. {\bf{Tambi\'en}} los deben llevar casi
terminados. En la complementaria (adem\'as de resolver dudas) se
calificar\'a la participaci\'on al tratar de resolver estos ejercicios
finales en el tablero. La idea es poder terminar los  detalles de esos
\'ultimos puntos para entregarlos al final de la complementaria.
\end{itemize}

\begin{enumerate}

\item
Ejercicio 20.5 (Planta Nuclear.)

\item
Ejercicio 20.15 (Ciclo de Carnot.)

\item 
Ejercicio 20.22 (Máquina de Carnot que derrite hielo.)

\item 
Ejercicio 20.26 (Entropía en una bañera.)

\item 
Ejercicio 20.40 (Máquina térmica de tres pasos.)

\item 
Ejercicio 20.60 (Diagrama TS)


\item
Encuentre el cambio de entropía de $1$ kg de hielo al derretirse.

\item
Encuentre el cambio de entropía de una masa $m$ de gas ideal que pasaa de un volumen $V_1$ y temperatura $T_1$ a un volumen $V_2$ y temperature $T_2$ en los siguientes casos: a) Se caliente a volumen constante $V_1$ y luego se expande de manera isoterma; b) Se expande a temperatura constante $T_1$ hasta un volumen $V_2$ y luego se caliente a volumen constante; c) se expande adiabáticamente hasta el volumen $V_2$ y luego se calienta a volumen constante.

\item
Mostrar que el rendimiento de una máquina térmica es máximo en un proceso cíclico en el cual la entropía del sistema no varía.

\item 
¿Cuánto trabajo se puede producir si se toma un iceberg de un volumen $1$ km$^3$ por fuente fría de una máquina térmica y el océano por fuente caliente? ¿Cuánto tiempo le tomaría a una central nuclear de 6GW de potencia producir la misma cantidad de energía?

\end{enumerate}
\end{document}
