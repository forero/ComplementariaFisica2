\documentclass{article}
\textheight=25.5cm
\textwidth=16.0cm
\oddsidemargin=-0.5cm
\topmargin=-3.0cm
\usepackage[pdftex]{graphicx}
\usepackage[utf8]{inputenc}
\usepackage[spanish]{babel}
\title{Taller \# 6 de F\'isica 2\\ FISI 1028, Semestre 2014 - 10}
\author{Profesor: Jaime Forero}
\date{Mi\'ercoles, 26 de Febrero, 2014}
\begin{document}
\maketitle
\thispagestyle{empty}

\noindent

Este taller debe ser preprarado y discutido para la clase
complementaria de la semana del 3 de Marzo del 2014. 

\noindent
{\bf Importante}:
\begin{itemize}

\item
Las respuestas a los ocho primeros ejercicios se deben entregar al comenzar la
clase complementaria. 
\item 
Los \'ultimos cuatro problemas son para participaci\'on en clase y entrega
durante la complementaria. {\bf{Tambi\'en}} los deben llevar casi
terminados. En la complementaria (adem\'as de resolver dudas) se
calificar\'a la participaci\'on al tratar de resolver estos ejercicios
finales en el tablero. La idea es poder terminar los  detalles de esos
\'ultimos puntos para entregarlos al final de la complementaria.
\end{itemize}

\begin{enumerate}

\item
Ejercicio 21.33 (Electr\'on entre dos placas).

\item
Ejercicio 21.35 (Rapidez del electr\'on entre dos placas).

\item
Ejercicio 21.48 (Campo El\'ectrico de dos cargas puntuales).

\item
Ejercicio 21.57 (Campo El\'ectrico entre dos l\'aminas).

\item
Ejercicio 21.62 (L\'ineas de campo el\'ectrico de tres cargas).

\item 
Ejercicio 21.69 (Par de torsi\'on sobre un dipolo).

\item
Ejercicio 22.4 (Flujo de campo a trav\'es de un cubo).

\item 
Ejercicio 22.11 (Ley de Gauss en un medio homog\'eneo).

\item 
Problema 21.89 (carga puntual y varillla cargada I).

\item
Problema 21.90 (carga puntual y varilla cargada II).

\item 
Problema 21.96 (carga en el centro de una semi-anillo).

\item
Problema 21.103 (dos l\'aminas infinitas).
\end{enumerate}

\end{document}
