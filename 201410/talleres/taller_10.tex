\documentclass{article}
\textheight=25.5cm
\textwidth=16.0cm
\oddsidemargin=-0.5cm
\topmargin=-3.0cm
\usepackage[pdftex]{graphicx}
\usepackage[utf8]{inputenc}
\usepackage[spanish]{babel}
\title{Taller \# 10 de F\'isica 2\\ FISI 1028, Semestre 2014 - 10}
\author{Profesor: Jaime Forero}
\date{Mi\'ercoles, 26 de Marzo, 2014}
\begin{document}
\maketitle
\thispagestyle{empty}

\noindent

Este taller debe ser preprarado y discutido para la clase
complementaria de la semana del 31 de Marzo del 2014. 

\noindent
{\bf Importante}:
\begin{itemize}

\item
Las respuestas a los {\bf cinco primeros} ejercicios se deben entregar
al comenzar la clase complementaria. 
\item 

Los \'ultimos siete problemas son para participaci\'on en clase y entrega
durante la complementaria. {\bf{Tambi\'en}} los deben llevar casi
terminados. En la complementaria (adem\'as de resolver dudas) se
calificar\'a la participaci\'on al tratar de resolver estos ejercicios
finales en el tablero. La idea es poder terminar los  detalles de esos
\'ultimos puntos para entregarlos al final de la complementaria.
\end{itemize}

\begin{enumerate}

\item
Ejercicio 25.7 (Corriente como funci\'on del tiempo).

\item
Ejercicio 25.31 (Ca\'ida de potencial en un cable).

\item 
Ejercicio 25.34 (Lectura de un amper\'imetro ideal en paralelo).

\item 
Ejercicio 25.35 (Lectura de un volt\'iemtro ideal en serie).

\item 
Ejercicio 25.43 (Potencia en bombillas el\'ectricas).

\item
Ejercicio 26.1 (Resistencia de un alambre en tres piezas).

\item 
Ejercicio 26.6 (Circuito para simplificar)

\item
Ejercicio 26.21 (Reglas de Kirchhoff)

\item
Ejercicio 26.24 (Reglas de Kirchhoff)

\item
Ejercicio 26.26 (Reglas de Kirchhoff)

\item
Ejercicio 26.29 (Reglas de Kirchhoff)

\item 
Ejercicio 26.41 (Circuito RC)

\end{enumerate}

\end{document}
