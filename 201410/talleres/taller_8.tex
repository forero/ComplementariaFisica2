\documentclass{article}
\textheight=25.5cm
\textwidth=16.0cm
\oddsidemargin=-0.5cm
\topmargin=-3.0cm
\usepackage[pdftex]{graphicx}
\usepackage[utf8]{inputenc}
\usepackage[spanish]{babel}
\title{Taller \# 8 de F\'isica 2\\ FISI 1028, Semestre 2014 - 10}
\author{Profesor: Jaime Forero}
\date{Mi\'ercoles, 12 de Marzo, 2014}
\begin{document}
\maketitle
\thispagestyle{empty}

\noindent

Este taller debe ser preprarado y discutido para la clase
complementaria de la semana del 17 de Marzo del 2014. 

\noindent
{\bf Importante}:
\begin{itemize}

\item
Las respuestas a los {\bf cuatro primeros} ejercicios se deben entregar
al comenzar la clase complementaria. 
\item 

Los \'ultimos dos problemas son para participaci\'on en clase y entrega
durante la complementaria. {\bf{Tambi\'en}} los deben llevar casi
terminados. En la complementaria (adem\'as de resolver dudas) se
calificar\'a la participaci\'on al tratar de resolver estos ejercicios
finales en el tablero. La idea es poder terminar los  detalles de esos
\'ultimos puntos para entregarlos al final de la complementaria.
\end{itemize}

\begin{enumerate}


\item
Ejercicio 23.22 (Potenial de dos cargas positivas).

\item 
Ejercicio 23.31 (Electron acelerado).

\item 
Ejercicio 23.42 (Equipotenciales entre dos placas).

\item
Ejercicio 23.48 (Campo a partir del potencial).

\item
Ejercicio 23.51 (Equipotenciales de un cilindro).

\item 
Problema 23.57 (Cristal i\'onico).

\end{enumerate}

\end{document}
