\documentclass{article}
\textheight=25.5cm
\textwidth=16.0cm
\oddsidemargin=-0.5cm
\topmargin=-3.0cm
\usepackage[pdftex]{graphicx}
\usepackage[utf8]{inputenc}
\usepackage[spanish]{babel}
\title{Taller \#2 de F\'isica 2\\ FISI 1028, Semestre 2014 - 10}
\author{Profesor: Jaime Forero}
\date{Mi\'ercoles, 29 de Enero, 2014}
\begin{document}
\maketitle
\thispagestyle{empty}

\noindent
Este taller debe ser preprarado y discutido para la clase
complementaria de la semana del 3 de Febrero del 2014. 

\noindent
{\bf Importante}:
\begin{itemize}

\item
Las respuestas a los seis primeros ejercicios se deben entregar al comenzar la
clase complementaria. 
\item 
Los \'ultimos cinco ejercicios son para participaci\'on en clase y entrega
durante la complementaria. {\bf{Tambi\'en}} los deben llevar casi
terminados. En la complementaria (adem\'as de resolver dudas) se
calificar\'a la participaci\'on al tratar de resolver estos ejercicios
finales en el tablero. La idea es poder terminar los  detalles de esos
\'ultimos puntos para entregarlos al final de la complementaria.
\end{itemize}

\begin{enumerate}

 

\item Ejercicio 18.22. (Masa molar de un compuesto org\'anico).

\item Ejercicio 18.39 (Rapidez rms de nitr\'ogeno y hidr\'ogeno).

\item Ejercicio 18.50 (Nubes)

\item Ejercicio 19.1 (Gr\'afica PV y trabajo)

\item Ejercicio 19.8 (Energ\'ia interna y primera ley)

\item Ejercicio 19.17 (Trabajo en procesos c\'iclicos) (Problema del
  primer parcial del 2013-20).

\item El volumen de un gas se duplica de dos maneras diferentes. Una
  vez lo hace de manera isoterma. Otra vez lo hace de manera
  isobara. ¿En cuál de los dos casos el gas hace el trabajo m\'as grande?

\item 
Considere dos recipientes de un mismo volumen cada uno conteniendo las
mismas moles de un gas ideal monoat\'omico. La presi\'on en un
recipiente es $P_{0}$ mientras que en el otro recipiente es
$4P_{0}$. Si la velocidad $v_{\rm rms}$ de los \'atomos en el primer
recipiente es $v_{0}$, ¿cuánto es la velocidad $v_{\rm rms}$ en el segundo
recipiente? (problema del primer parcial del 2013-20)

\item
Estime el n\'umero de mol\'eculas de aire que golpean 1 cm$^2$ de muro
de su habitaci\'on en un segundo y el impulso (impulso $=\Delta p/
\Delta t$, donde $p$ es momentum) que le transmiten al muro.

\item 
¿Por qu\'e los bombillos se llenan de un gas inerte con una presi\'on
mucho menor a la presi\'on atmosf\'erica?

\item 
Para medir la masa de agua que contienen las gotas de neblina se
encierra herm\'eticamente una muestra de aire a una presion de
$100$kPa a una temperatura de $0^{\circ}$C en un recipiente
herm\'etico a paredes transparentes. Luego se calienta hasta que la
neblina desaparezca y se mide la presi\'on a esta temperatura. Calcule
la masa de neblina dentro de 1$m^3$ de muestra si la temperatura a la
cual desparece la neblina es de $82^{\circ}$ y la presi\'on del aire
a esta temperatura es de $180$kPa. (Pista: la respuesta debería ser menor
800 gr y mayor que 50 gr.)

\end{enumerate}
\end{document}
