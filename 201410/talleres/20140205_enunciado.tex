\documentclass{article}
\textheight=25.5cm
\textwidth=16.0cm
\oddsidemargin=-0.5cm
\topmargin=-3.0cm
\usepackage[pdftex]{graphicx}
\usepackage[utf8]{inputenc}
\usepackage[spanish]{babel}
\title{Taller \#3 de F\'isica 2\\ FISI 1028, Semestre 2014 - 10}
\author{Profesor: Jaime Forero}
\date{Mi\'ercoles, 5 de Febrero, 2014}
\begin{document}
\maketitle
\thispagestyle{empty}

\noindent
Este taller debe ser preprarado y discutido para la clase
complementaria de la semana del 10 de Febrero del 2014. 

\noindent
{\bf Importante}:
\begin{itemize}

\item
Las respuestas a los seis primeros ejercicios se deben entregar al comenzar la
clase complementaria. 
\item 
Los \'ultimos cuatro ejercicios son para participaci\'on en clase y entrega
durante la complementaria. {\bf{Tambi\'en}} los deben llevar casi
terminados. En la complementaria (adem\'as de resolver dudas) se
calificar\'a la participaci\'on al tratar de resolver estos ejercicios
finales en el tablero. La idea es poder terminar los  detalles de esos
\'ultimos puntos para entregarlos al final de la complementaria.
\end{itemize}

\begin{enumerate}

\item
Ejercicio 19.20 (Trabajo en una compresi\'on isot\'ermica.)

\item
Ejercicio 19.25 (Fracci\'on de calor transferido que hace trabajo.)

\item 
Ejercicio 19.35 (Energ\'ia interna en un proceso adiab\'atico.)

\item 
Ejercicio 19.39 (Expansi\'on adiab\'atica de aire.)

\item 
Ejercicio 19.43 (Ciclos en un sistema arbitrario).

\item 
Ejercicio 19.51 (Cambio de volumen en un sistema no ideal).


\item
¿Cuál es la energía interna (en Joules) de 1 cm$^3$ de aire en condiciones normales de presi\'on y temperatura?

\item
Para subir de $1$ K la temperatura de $1$ kg de un gas desconocido a
presi\'on constante se necesitan $912$ J. Para calentarlo a volumen
constante se necesitan $649$ J. ¿Cuál es el gas desconocido?

\item
Un pist\'on de masa $M$ tiene encerrado un volumen $V_0$ de un gas
monoat\'omico a presi\'on $P_0$ y temperatura $T_0$. El pist\'on se
mueve inicialmente con una velocidad $u$ y comprime el gas. Suponiendo
que no hay fugas de calor y despreciando las capacidades calor\'ificas
del pistón y el recipiente, determine la temperatura y el volumen del gas en su
compresi\'on m\'axima.

\item 
Considere una mol de un gas monoat\'omico ideal a temperatura $T_0$
que se encuentra dentro de un tubo largo entre dos pistones de masa
$m$. En un momento inicial los pistones tienen velocidades iguales a
$3v$ y $v$ que van en la misma direcci\'on. Las velocidades de los pistones son
tales que el gas comienza a comprimirse. Demuestre que la
temperatura máxima que alcanza el gas es 
\begin{displaymath}
T_{\rm max} = T_{0} + \frac{2mv^2}{3R}.
\end{displaymath}
Considere que los pistones no conducen el calor y que el gas tiene una
masa despreciable con respecto a la masa total del gas.  Tambi\'en
desprecie las capacidades calor\'ificas de los pistones y el recipiente.



\end{enumerate}
\end{document}
