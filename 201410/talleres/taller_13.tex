\documentclass{article}
\textheight=25.5cm
\textwidth=16.0cm
\oddsidemargin=-0.5cm
\topmargin=-3.0cm
\usepackage[pdftex]{graphicx}
\usepackage[utf8]{inputenc}
\usepackage[spanish]{babel}
\title{Taller \# 13 de F\'isica 2\\ FISI 1028, Semestre 2014 - 10}
\author{Profesor: Jaime Forero}
\date{Mi\'ercoles, 30 de Abril, 2014}
\begin{document}
\maketitle
\thispagestyle{empty}

\noindent

Este taller debe ser preprarado y discutido para la clase
complementaria de la semana del 5 de Mayo del 2014. 

\noindent
{\bf Importante}:
\begin{itemize}

\item
Las respuestas a los {\bf cinco primeros} ejercicios se deben entregar
al comenzar la clase complementaria. 
\item 

Los \'ultimos cuatro problemas son para participaci\'on en clase y entrega
durante la complementaria. {\bf{Tambi\'en}} los deben llevar casi
terminados. En la complementaria (adem\'as de resolver dudas) se
calificar\'a la participaci\'on al tratar de resolver estos ejercicios
finales en el tablero. La idea es poder terminar los  detalles de esos
\'ultimos puntos para entregarlos al final de la complementaria.
\end{itemize}

\begin{enumerate}

\item
Ejercicio 29.7 (Alambre y espira).

\item
Ejercicio 29.21 (Alambre en movimiento).

\item
Ejercicio 29.25 (Varilla sobre rieles met\'alicos).

\item
Ejercicio 30.1 (Inductancia mutua de dos bobinas).

\item
Ejercicio 30.11 (Inductancia de un Solenoide).

\item
Problema 29.56 (Rapidez terminal).

\item
Problema 29.58 (FEM en una bala).

\item
Problema 29.61 (Fem inducida sobre un alambre y una espira).

\item
Problema 29.65 (Fuerza sobre alambre corredizo).


\end{enumerate}

\end{document}
