\documentclass{article}
\textheight=25.5cm
\textwidth=16.0cm
\oddsidemargin=-0.5cm
\topmargin=-3.0cm
\usepackage[pdftex]{graphicx}
\usepackage[utf8]{inputenc}
\usepackage[spanish]{babel}
\title{Taller \# 12 de F\'isica 2\\ FISI 1028, Semestre 2014 - 10}
\author{Profesor: Jaime Forero}
\date{Mi\'ercoles, 9 de Abril, 2014}
\begin{document}
\maketitle
\thispagestyle{empty}

\noindent

Este taller debe ser preprarado y discutido para la clase
complementaria de la semana del 21 de Abril del 2014. 

\noindent
{\bf Importante}:
\begin{itemize}

\item
Las respuestas a los {\bf cinco primeros} ejercicios se deben entregar
al comenzar la clase complementaria. 
\item 

Los \'ultimos cuatro problemas son para participaci\'on en clase y entrega
durante la complementaria. {\bf{Tambi\'en}} los deben llevar casi
terminados. En la complementaria (adem\'as de resolver dudas) se
calificar\'a la participaci\'on al tratar de resolver estos ejercicios
finales en el tablero. La idea es poder terminar los  detalles de esos
\'ultimos puntos para entregarlos al final de la complementaria.
\end{itemize}

\begin{enumerate}

\item
Ejercicio 28.4 (Campo producido por una part\'icula alfa y  un electr\'on).

\item
Ejercicio 28.31 (Campo de un alambre en \'angulo recto).

\item 
Ejercicio 28.15 (Campo magn\'etico de un rel\'ampago).

\item 
Ejercicio 28.20 (Efecto de las l\'ineas de transmisi\'on).

\item
Ejercicio 28.28 (Fuerza entre tres alambres paralelos).

\item 
Ejercicio 28.56 (Configuraciones de campo magn\'etico cero).

\item
Ejercicio 28.68 (Campo de una espira).

\item 
Ejercicio 28.76 (Campo de una espira y un alambre).

\item
Ejercicio 28.81 (L\'amina infinita de corriente).

\end{enumerate}

\end{document}
