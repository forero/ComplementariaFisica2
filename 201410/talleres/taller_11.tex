\documentclass{article}
\textheight=25.5cm
\textwidth=16.0cm
\oddsidemargin=-0.5cm
\topmargin=-3.0cm
\usepackage[pdftex]{graphicx}
\usepackage[utf8]{inputenc}
\usepackage[spanish]{babel}
\title{Taller \# 11 de F\'isica 2\\ FISI 1028, Semestre 2014 - 10}
\author{Profesor: Jaime Forero}
\date{Mi\'ercoles, 2 de Abril, 2014}
\begin{document}
\maketitle
\thispagestyle{empty}

\noindent

Este taller debe ser preprarado y discutido para la clase
complementaria de la semana del 7 de Abril del 2014. 

\noindent
{\bf Importante}:
\begin{itemize}

\item
Las respuestas a los {\bf cinco primeros} ejercicios se deben entregar
al comenzar la clase complementaria. 
\item 

Los \'ultimos cuatro problemas son para participaci\'on en clase y entrega
durante la complementaria. {\bf{Tambi\'en}} los deben llevar casi
terminados. En la complementaria (adem\'as de resolver dudas) se
calificar\'a la participaci\'on al tratar de resolver estos ejercicios
finales en el tablero. La idea es poder terminar los  detalles de esos
\'ultimos puntos para entregarlos al final de la complementaria.
\end{itemize}

\begin{enumerate}

\item
Ejercicio 27.1 (Fuerza sobre una part\'icula).

\item
Ejercicio 27.10 (Flujo magn\'etico en un cubo).

\item 
Ejercicio 27.15 (Tiempo para dar media vuelta en el campo magn\'etico).

\item 
Ejercicio 27.19 (Reactor de Fusi\'on).

\item
Ejercicio 27.30 (Campos magn\'eticos y el\'ectricos transversales)

\item 
Ejercicio 27.40 (Balanza magn\'etica)
\item
Ejercicio 27.44 (Bobina rectangular)

\item
Ejercicio 27.72 (Cañon magn\'etico de rieles)

\item
Ejercicio 27.67 (Alambre en un plano inclinado)

\end{enumerate}

\end{document}
