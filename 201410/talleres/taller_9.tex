\documentclass{article}
\textheight=25.5cm
\textwidth=16.0cm
\oddsidemargin=-0.5cm
\topmargin=-3.0cm
\usepackage[pdftex]{graphicx}
\usepackage[utf8]{inputenc}
\usepackage[spanish]{babel}
\title{Taller \# 9 de F\'isica 2\\ FISI 1028, Semestre 2014 - 10}
\author{Profesor: Jaime Forero}
\date{Mi\'ercoles, 19 de Marzo, 2014}
\begin{document}
\maketitle
\thispagestyle{empty}

\noindent

Este taller debe ser preprarado y discutido para la clase
complementaria de la semana del 24 de Marzo del 2014. 

\noindent
{\bf Importante}:
\begin{itemize}

\item
Las respuestas a los {\bf cuatro primeros} ejercicios se deben entregar
al comenzar la clase complementaria. 
\item 

Los \'ultimos cinco problemas son para participaci\'on en clase y entrega
durante la complementaria. {\bf{Tambi\'en}} los deben llevar casi
terminados. En la complementaria (adem\'as de resolver dudas) se
calificar\'a la participaci\'on al tratar de resolver estos ejercicios
finales en el tablero. La idea es poder terminar los  detalles de esos
\'ultimos puntos para entregarlos al final de la complementaria.
\end{itemize}

\begin{enumerate}

\item
Ejercicio 24.4 (Capacitancia de un osciloscopio)

\item
Ejercicio 24.27 (Energ\'ia el\'ectrica convertida en energ\'ia t\'ermica.)

\item
Ejercicio 24.29 (Energ\'ia almacenada en un capacitor.)

\item
Ejercicio 24.38 (Capacitor con un diel\'ectrico)

\item
Ejercicio 24.66 (Capacitor con un bloque met\'alico adentro)

\item ¿Con qu\'e fuerza se atraen dos placas paralelas que tienen densidades superficiales de carga opuestas? Suponga que la densidad superficial es $\pm \sigma$, el \'area de cada placa es $A$ y la distancia entre las placas es mucho menor que sus dimensiones. ¿Qué fuerza por unidad de área se siente en cada placa (esto correspondería a una especie de presión eléctrica)?

\item  (Problema del tercer parcial del 2013-20) El potencial de un conductor cargado es igual a
  300V. ¿Cual debe ser la velocidad m\'inima de un electron para poder
  alejarse de la superficie del conductor hasta el infinito?
\begin{enumerate}
\item[a)] $1.0\times 10^7$ m/s
\item[b)] $1.0\times 10^7$ cm/s
\item[c)] $1.0\times 10^{14}$ m/s
\item[d)] $1.0\times 10^{-5}$ m/s
\end{enumerate}


\item (Problema del tercer parcial del 2013-20) Cuanto vale la diferencia de potencial en las placas
  extremas de de un sistema compuesto por tres planos infinitos
  paralelos cargados con densidades de carga $\sigma_1$, $\sigma_2$ y
  $\sigma_3$? La placa intermedia se encuentra a una distancia
  $h_1$de la primera y a una distancia $h_2$ de la tercera. 

\begin{enumerate}
\item[a)] $\frac{1}{2\epsilon_0} (\sigma_1 -\sigma_2 -\sigma_3) h_1$ 
\item[b)] $\frac{1}{2\epsilon_0} (\sigma_1 +\sigma_2 + \sigma_3)
  (h_1+h_2)$ 
\item[c)] $\frac{1}{2\epsilon_0}((h_1 + h_2) (\sigma_3-\sigma_1) +
  \sigma_2(h_2-h_1))$ 
\item[d)] $\frac{1}{2\epsilon_0} ((h_1 + h_2)(\sigma_1 + \sigma_2) +
  \sigma_3(h_2-h_1))$ 
\end{enumerate}


\item (Problema del tercer parcial del 2013-20) Una bola met\'alica de radio $10$cm se encuentra al
  interior de un caparaz\'on esf\'erico de radio exterior $30$cm y
  $10$cm de grosor. Los centros de la bola y el caparaz\'on coinciden.
  La bola tiene una carga de $10^{-5}$ C y el caparaz\'on tiene una
  carga de $8\times 10^{-5}$ C. Construya una gr\'afica de la
  variaci\'on del potencial como funci\'on de la distancia al centro
  de la bola. Los ejes deben marcar adecuadamente las cantidades en
  Volts y en cent\'imetros.  



\end{enumerate}

\end{document}
