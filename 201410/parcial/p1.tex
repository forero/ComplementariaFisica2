\documentclass{article}
\addtolength{\oddsidemargin}{-.875in}
\addtolength{\evensidemargin}{-.875in}
\addtolength{\textwidth}{1.75in}
\addtolength{\topmargin}{-.875in}
\addtolength{\textheight}{1.70in}
\usepackage[pdftex]{graphicx}
\usepackage[utf8]{inputenc}
\usepackage[spanish]{babel}
\title{Parcial \#1  - Física II (2014-10)\\ }
\author{Profesor: Jaime Forero}
\date{19 de febrero del 2014}
\begin{document}
\maketitle


{\bf Instrucciones}\\
\begin{itemize}
\item
Lea completo el ex\'amen antes de empezar a responderlo. Hay 2 p\'aginas y 8 ejercicios. Pregunte en voz alta si tiene dudas sobre alg\'un enunciado.
\item
El ex\'amen tiene 110 puntos posibles  como m\'aximo y 0.0 puntos como m\'inimo. 100 puntos corresponden a una nota de 5.0
\item 
Todas las respuestas, incluso a los ejercicios de selecci\'on m\'ultiple, deben incluir una justificaci\'on. Solamente se otorgan los puntos completos de las preguntas de selecci\'on m\'ultiple si la respuesta y la justificaci\'on son correctas. En cualquier otro caso se otorgan cero (0) puntos.
\item
Para escribir las respuestas y  los c\'alculos auxiliares, solamente se pueden utilizar las hojas blancas que reciben al principio. Cada hoja debes estar marcada con nombre y c\'odigo. Las hojas de borrador se deben entregar al final.
\item SI se permite el uso de calculadora.
\item
El uso de tel\'efonos celulares, laptops, tabletas, etc durante un examen ser\'a considerado como fraude. Esta falta ser\'a reportada ante el comit\'e disciplinario de la Facultad de Ciencias para entablar el proceso correspondiente seg\'un el reglamento de pregrado. 
\end{itemize}


\input{formulas}
\newpage
\begin{enumerate}

\item (10 puntos) 
Un gas con temperatura uniforme está confinado en una mitad de una
cámara doble aislada (no hay transferencia de energía con el ambiente)
a través de una membrana. La otra mitad de la cámara se evacúa
(quedando solo vacío). Si la membrana se rompe y el gas se expande
libremente para llenar toda la doble cámara, cuál de las siguientes
afirmaciones acerca del proceso es correcta. 

\begin{itemize}
\item[a)] $W>0$, $Q=0$, $\Delta U<0$
\item[b)] $W<0$, $Q<0$, $\Delta U=0$
\item[c)] $W=0$, $Q<0$, $\Delta U<0$
\item[d)] $W=0$, $Q=0$, $\Delta U=0$
\end{itemize}

\item (10 puntos)
El volumen de un gas se disminuye a la mitad y su temperatura aumenta $1.5$ veces. ¿Cuánto aumentó la presión del gas?
\begin{itemize}
\item[a)]  3 veces
\item[b)]  1/3 veces
\item[c)]  4/3 veces
\item[d)]  3/4 veces.
\end{itemize}


\item (10 puntos) Para enfriar $400$ g de agua caliente a 70 $^\circ$C,
  se agrega un pedazo de hielo de $10$ g a $-10$
  $^{\circ}$C. Despu\'es de que el sistema llega al equilibrio ¿Se
  derrite todo el hielo? El calor espec\'ifico
  del agua l\'iquida es $4190$ J kg$^{-1}$ K$^{-1}$, del agua
  s\'olida es $2144$ J kg$^{-1}$ K$^{-1}$ y el calor latente de
  fusi\'on del agua es $334000$ J/kg. 
\begin{itemize}
\item[a)] S\'i
\item[b)] No
\end{itemize}

\item (15 puntos) ¿Cuánto var\'ian entre s\'i las velocidades $rms$ de
  las mol\'eculas en dos recipientes diferentes que contienen gases a
  la misma temperatura, presión, volumen y cantidad de moles, pero
  el gas en el segundo recipiente pesa 6 veces más que en el primero?

\item (15 puntos) ¿Cuánto vale \emph{aproximadamente} la temperatura
  del hilo de un bombillo incandescente que emite energía con una
  potencia de 150 W? Asuma que el calor se transmite por radiación
  únicamente. 

\item (15 puntos)
Cu\'al es la energ\'ia interna (en Joules) de $1$ litro de aire en
condiciones normales de presi\'on y temperatura ($1$ atm=$10^5$ Pa).
Asuma que el aire est\'a compuesto de nitr\'ogeno molecular $N_2$ y
que la masa molar del Nitr\'ogeno monoat\'omico es $7$g.


\item (15 puntos) Un recipiente de volumen fijo contiene un gas ideal
  monoat\'omico con una 1 mol de gas a temperatura $400$K. Un segundo
  recipiente con el mismo volumen tiene el doble de moles de un gas
  ideal diat\'omico. La presi\'on en ambos recipientes es la
  misma. Cada uno de estos recipientes recibe el mismo flujo de calor
  constante y su temperatura empieza a aumentar. ¿Cu\'al de los dos
  gases alcanza primero la temperatura $1000$K?  



\item (20 puntos) Una cantidad $Q$ de calor fluye hacia un gas ideal con
  exponente adiab\'atico $\gamma$. ¿Qu\'e fracc\'ion de la energ\'ia
  calor\'ifica es utilizada para efectuar el trabajo de expansi\'on
  del gas, asumiendo que el proceso ocurre a presi\'on constante? Es
  decir, ¿cu\'anto vale $W/Q$ para este proceso? La respuesta debe
  estar expresada en t\'erminos de $\gamma$ \'unicamente. 

\end{enumerate}



%\begin{figure}[h!]
%\begin{center}
%\includegraphics[scale=0.40]{PV-plot.pdf}
%\end{center}
%\caption{Diagrama para el problema \ref{prob:PV}.}
%\label{fig:PV}
%\end{figure}







\end{document}
