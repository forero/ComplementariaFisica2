\documentclass{article}
\addtolength{\oddsidemargin}{-.875in}
\addtolength{\evensidemargin}{-.875in}
\addtolength{\textwidth}{1.75in}
\addtolength{\topmargin}{-.875in}
\addtolength{\textheight}{1.70in}
\usepackage[pdftex]{graphicx}
\usepackage[utf8]{inputenc}
\usepackage[spanish]{babel}
\title{Parcial \#2  - Física II (2014-10)\\ }
\author{Profesor: Jaime Forero}
\date{14 de marzo del 2014}
\begin{document}
\maketitle


{\bf Instrucciones}\\
\begin{itemize}
\item
Lea completo el ex\'amen antes de empezar a responderlo. Hay 3 p\'aginas y 9 ejercicios. Pregunte en voz alta si tiene dudas sobre alg\'un enunciado.
\item
El ex\'amen tiene 120 puntos posibles  como m\'aximo y 0.0 puntos como m\'inimo. 100 puntos corresponden a una nota de 5.0
\item 
Todas las respuestas, incluso a los ejercicios de selecci\'on m\'ultiple, deben incluir una justificaci\'on. Solamente se otorgan los puntos completos de las preguntas de selecci\'on m\'ultiple si la respuesta y la justificaci\'on son correctas. En cualquier otro caso se otorgan cero (0) puntos.
\item
Para escribir las respuestas y  los c\'alculos auxiliares, solamente se pueden utilizar las hojas blancas que reciben al principio. Cada hoja debes estar marcada con nombre y c\'odigo. Las hojas de borrador se deben entregar al final.
\item SI se permite el uso de calculadora.
\item
El uso de tel\'efonos celulares, laptops, tabletas, etc durante un examen ser\'a considerado como fraude. Esta falta ser\'a reportada ante el comit\'e disciplinario de la Facultad de Ciencias para entablar el proceso correspondiente seg\'un el reglamento de pregrado. 
\end{itemize}


\input{formulas}


\input{formulas_electro}

\newpage
\begin{enumerate}


\item (10 puntos) 
Tenemos dos pedazos de metal cada uno con la misma masa $m$ en kilogramos. El
primero se encuentra a una temperatura de $27^{\circ}$C y el segundo a
$47^{\circ}$C. Los dos pedazos de metal se dejan en contacto hasta que
llegan al equilibrio t\'ermico. Todo esto sucede dentro de un
recipiende aislado del ambiente para evitar p\'erdidas de calor. Si el
calor específico es $c$ en unidades de J kg$^{-1}$ K$^{-1}$ ¿Cuánto es el cambio de entropía total de este
sistema aislado?  

\begin{itemize}
\item[a)]  $mc/2$ J/K
\item[b)]  0 J/K
\item[c)]  $mc$ ln $(961/960)$ J/K
\item[d)]  $mc$ ln $(1369/1269)$ J/K
\item[e)]  ninguna de las anteriores.
\end{itemize}


\item (10 puntos) Una m\'aquina de Carnot trabaja entre una fuente
  caliente a temperatura de $4000$K y una fuente fr\'ia a $200$K. La
  máquina toma $400$J de la fuente caliente. ¿Cuánto es
  el trabajo que produce esta máquina en cada ciclo? \\


a) 240 J b) 400 J  c) 380 J d) 20 J e) ninguna de las anteriores\\


\item (10 puntos) A partir de un alambre cargado positivamente con
  densidad lineal de carga $+\lambda$ y otro alambre cargado
  negativamente con densidad lineal de carga $-\lambda$ se construyen
  las dos configuraciones mostradas en la Figura. ¿Cómo se compara la
  magnitud del campo eléctrico en estos dos casos?

\begin{itemize}
\item[a)] El campo es cero en ambos casos.
\item[a)] El campo es mayor en la primera configuraci\'on.
\item[b)] El campo es mayor en la segunda configuraci\'on.
\item[c)] El campo es diferente de cero y de igual magnitud en ambos casos.
\end{itemize}

\item (10 puntos) En la vecindad de
  una carga puntual $-Q$ se encuentran varios dipolos eléctricos de
  cargas mucho más pequeñas que $Q$. Los dipolos se mantien en su
  lugar por un alfiler que pasa por su centro de masa. Asumiendo la convenci\'on donde
  la flecha de un dipolo apunta  de la carga positiva a la carga
  negativa, ¿cómo se deber\'ian alinear los dipolos cuando se
  encuentran en equilibrio estable? Marque su respuesta en la Figura.

\item (10 puntos) Se tienen dos cargas iguales de valor $q$ y masa
  $m$ colgando como dos p\'endulos de longiud $l$. ¿Cuál el valor del
  ángulo $\alpha$ formado entre ellas dos? Exprese su respuesta en
  t\'erminos de $l$, $m$, $q$, $\epsilon_0$ y  la aceleraci\'on de la
  gravedad $g$. 

\item Dos planos infinitos con carga superficial $\sigma$
  se intersectan como muestra la Figura. 
  \begin{enumerate}
\item[a)] (5 puntos) Dibuje en la misma Figura la
  direcci\'on del campo neto en las dos regiones indicadas 
\item[b)] (10 puntos) Calcule la intensidad del campo en t\'erminos de $\sigma$,
  $\epsilon_0$ y el \'angulo $\beta$ en las dos regiones indicadas.
\end{enumerate}

\item (15 puntos) Considere una esfera de radio $R$ con una carga total $2Q$ que
  est\'a distribu\'ida de manera homog\'enea en su volumen. Adentro de esta esfera
  se encuentran dos cargas puntuales, cada una con carga $-Q$. Las
  cargas se encuentran sobre una l\'inea que pasa por el centro de la
  esfera. ¿Cuál es la distancia, $a$, que debe haber entre estas dos cargas
  puntuales para que la configuración se encuentre en equilibrio?
  Exprese su respuesta en t\'erminos de $R$ y de $Q$.

\item (20 puntos) Considere un cilindro infinito de radio $R$ y
  densidad volum\'etrica de carga $\rho$. Encuentre la direcci\'on y
  la intensidad del campo el\'ectrico como funci\'on de la distancia
  al eje de simetr\'ia del cilindro.

{\bf Bono}

\item (20 puntos) Considere ahora que en el punto anterior hay varias
  cargas negativas $-q$ de masa $m$ que entran con una velocidad $v_0$
  paralela al eje del cilindro a diferentes distancias del eje de
  simetr\'ia. Teniendo en cuenta el resultado del punto anterior
  describa de manera cuantitativa las trayectorias que describen las
  cargas negativas adentro del cilindro.  


\end{enumerate}



%\begin{figure}[h!]
%\begin{center}
%\includegraphics[scale=0.40]{PV-plot.pdf}
%\end{center}
%\caption{Diagrama para el problema \ref{prob:PV}.}
%\label{fig:PV}
%\end{figure}







\end{document}
