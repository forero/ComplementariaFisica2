\documentclass{article}
\textheight=25.5cm
\textwidth=16.0cm
\oddsidemargin=-0.5cm
\topmargin=-3.0cm
\usepackage[pdftex]{graphicx}
\usepackage[utf8]{inputenc}
\usepackage[spanish]{babel}
\title{Taller \#1 de F\'isica 2\\ FISI 1028, Semestre 2014 - 10}
\author{Profesor: Jaime Forero}
\date{Mi\'ercoles, 22 de Enero, 2014}
\begin{document}
\maketitle
\thispagestyle{empty}

\noindent
Este taller debe ser preprarado y discutido para la clase
complementaria de la semana del 27 de Enero del 2014. 

\noindent
{\bf Importante}:
\begin{itemize}
\item
Las respuestas a los seis primeros ejercicios se deben entregar al comenzar la
clase complementaria. 
\item 
Los \'ultimos cuatro ejercicios son para participaci\'on en clase y entrega
durante la complementaria. {\bf{Tambi\'en}} los deben llevar casi
terminados. En la complementaria (adem\'as de resolver dudas) se
calificar\'a la participaci\'on al tratar de resolver estos ejercicios
finales en el tablero. La idea es poder terminar los  detalles de esos
cuatro \'ultimos puntos para entregarlos al final de la complementaria.
\end{itemize}

\begin{enumerate}
\item{Ejercicio 17.6. Conversiones b\'asicas de temperatura.}
\item{Ejercicio 17.11. Temperatura del nitr\'ogeno l\'iquido.}
\item{Ejercicio 17.28. Expansi\'on t\'ermica de un pist\'on.} 
\item{Ejercicio 17.34. Calentando caf\'e.}
\item{Ejercicio 17.44. Estimar calores latentes de fusi\'on.}
\item{Ejercicio 17.52. Quemaduras de vapor vs. quemaduras de agua.}

\item La evaporaci\'on del sudor es un mecanismo
  importante para bajar la temperatura de algunos animales. ¿Qué masa
  de agua es mayor, la que requiere para evaporarse en la piel de un
  caballo de 500 kg para enfríar su cuerpo de  50$^{\circ}$C a
  30$^{\circ}$? ¿O la que se requiere para enfriar un caballo de 300 kg
  de 50$^{\circ}$C a 15$^{\circ}$? Asuma que el calor de
  vaporizaci\'on del agua y el calor espec\'ifico de un caballo son
  constantes entre 0$^{\circ}$C y 80$^{\circ}$. (Ejercicio tomado del primer
  parcial del 2013-20.)  

\item{Una tetera de aluminio de $2.0$kg que contiene $1.0$kg de agua se pone
en una estufa. Pensando que no se transfiere calor al entorno. ¿Cuánto
calor debe agregarse para elevar la temperatura de 20.0$^{\circ}$C a
100$^{\circ}$? Tome el calor espec\'ifico del Alumino como
$C_{Al}=900$ J kg$^{-1}$ K$^{-1}$ y el calor espec\'ifico del agua
como $C_{H20}=4200$ J kg$^{-1}$ K$^{-1}$. (Ejercicio tomado del primer parcial
del 2013-20.)}

\item{Si el sal\'on B202 estuviera aislado t\'ermicamente estime
  cuanto deber\'ia aumentar la temperatura durante una hora de clase
  de F\'isica II si todos los alumnos est\'an presentes. Pistas para
  la soluci\'on: ¿Cu\'antos
  Watts produce un estudiante promedio? ¿Cu\'anto es la masa de aire dentro
  del sal\'on? \footnote{Este es un ejemplo de lo que se conoce como
    Problemas de Fermi:
    {\texttt{http://en.wikipedia.org/wiki/Fermi\_problem}}. Durante el
    curso vamos a plantear varios problemas de este estilo}} 


\item{La cantidad de energ\'ia
  t\'ermica que puede ser almacenada por un ser vivo es proporcional a su
  masa. Por otro lado, las p\'erdidas de calor son proporcionales al
  área de la superficie del cuerpo, dado que estas p\'erdidas se dan en su
  mayor\'ia a trav\'es de intercambios con la piel. Pensemos ahora en
  un rat\'on y una vaca en el Polo Norte (¡?). Asumiendo que la
  composici\'on de ambos es la misma, y que inicialmente tienen la
  misma cantidad de energ\'ia t\'ermica por 
  unidad de masa. ¿Quien va a sufrir primero una muerte por hipotermia:
  el rat\'on o la vaca? Estime que tanto m\'as r\'apido se
  muere el uno que el otro \footnote{Este es un ejemplo de problemas que
    usan an\'alisis dimensional y leyes de escala (scaling laws). Tambi\'en
    tendremos varios de estos durante el semestre. Esta es una lectura
recomendada: {\texttt{http://galileo.phys.virginia.edu/classes/304/scaling.pdf}}}.} 
\end{enumerate}
\end{document}
